%\documentclass[conference]{IEEEtran}
\documentclass[twocolumn,twoside]{IEEEtran}

\IEEEoverridecommandlockouts
% The preceding line is only needed to identify funding in the first footnote. If that is unneeded, please comment it out.
% \usepackage{cite}
\usepackage{amsmath,amssymb,amsfonts}
\usepackage{algorithmic}
\usepackage{graphicx}
\usepackage{tabularx}
\usepackage{textcomp}
\usepackage{xcolor}
\usepackage{listings}

\usepackage[T1]{fontenc}
\usepackage{times}

%\usepackage{biblatex} %Imports biblatex package
%\addbibresource{bibliography.bib} %Import the bibliography file

\usepackage[numbers]{natbib}

\graphicspath{{images/}}


\def\BibTeX{{\rm B\kern-.05em{\sc i\kern-.025em b}\kern-.08em
    T\kern-.1667em\lower.7ex\hbox{E}\kern-.125emX}}

\begin{document}

\lstdefinestyle{mystyle}{
    basicstyle=\ttfamily\footnotesize,
    breakatwhitespace=false,    
    xleftmargin=4pt,     
    breaklines=true,                 
    captionpos=b,                    
    keepspaces=true,                 
    showspaces=false,                
    showstringspaces=false,
    showtabs=false,                  
    tabsize=2,
    keywordstyle=\color{black}\bfseries
}

\lstset{style=mystyle}



\title{
Recording the Lifecycle of Software Models
}

\author{\IEEEauthorblockN{Brendan P. Bonner}\\
\IEEEauthorblockA{\textit{School of Computing} \\
\textit{Dublin City University}\\
Dublin, Ireland \\
brendan.bonner2@mail.dcu.ie}

}

\maketitle

\begin{abstract}
Artificial Intelligence is built on trustworthiness of the systems that provide the intelligence. The foundation of this is built upon our understanding of the role of software models in this process. Similar to measuring social trust in humans, we can only increase our levels of trust by being exposed to the full competence and vulnerabilities of the environment to get to the point in time where trust can be assessed. The pathway for the human brain to grow and evolve is recorded in genealogy, education and experience. In this paper, we investigate the feasibility and benefits for replicating this process for machine learning systems. During this process, we developed an ecosystem where AI systems can have their entire lifecycle recorded and made available for scrutiny. The framework for a verifiable chain of provenance of the lifecycle of artificial intelligence development may be a first step into addressing shortcomings AI research has which potentially could stymie innovation in certain critical systems that need public support. The outcome of the research cannot enforce a system to be trustworthy, instead providing additional insight into a software models \textit{personality}, without needed to directly examine each neuron. 
\end{abstract}

\begin{IEEEkeywords}
Artificial Intelligence, Provenance, Explainability, Model Analysis, Trust, Cloud\end{IEEEkeywords}

\section{Introduction}

The topic covers the recording of CNN layer metrics during the training
sequence. The hypothesis is to examine trust mechanisms for explaining AI, and
providing an immutable record of the training sequences prior to a model being
utilised or customised.

The practicum focuses on two approaches to this;
recording and visualisation of the sequence of changes during the training of a
model. During model fitting at configurable intervals, layer characteristics are
measured and recorded, alongside mathematical verification, and stored. This can
be visualised, supporting modern explainable AI measures, to verify model
behaviour according to training, and can be trusted.

The proposal builds upon the attempts to provide additional metrics to support
the establishment of the concept of fairness in AI. In comparison with human
trust, which is established through verification and measurement of academic and
professional credentials, Fairness in AI is an addendum to the model being
examined, with attempts to quantify via measures such as Thiel scores for
certain bias tests. This is further developed by IBMs AI360 Fairness toolkit
paper which show how this is measured and corrected, but not being able to
identify where that behaviour was acquired and why a model is trained in a
certain way to develop these outcomes.
 There is currently a gap in identifying
if there is a provenance of trust when using deployed AI systems, and if sibling
and inherited models retain these behaviours and if this can be visualised.
While attempts to visualise the internals of decisions in the Simonyon and Bau
papers, these focus on trying to visualise the pathways being activated in a 2
dimensional map. The dissection and visualisation papers provide a key how to
measure and record this aspect of models, and our research will examine if there
is a potential to securely record neural network snapshots.
 
 The question being asked is to determine if we develop a novel mechanism and metric for
representing the evolution of knowledge within a neural network during the
training and development stages of creating a model. By analysing the state of
the art in terms of AI fairness and methods of visualising the dissected layers
common networks, the research will establish if creating an immutable chain of
layer delta changes over time will provide an insight into the trustworthiness
of the underlying model.

 The second part of the research will ask if the
information being stored can be visualised based on the elements of explainable
AI to be able to identify which iteration and epoch of the training process is
responsible for the manifestation of the behaviour. Ultimately, the outcome of
the research should ask if providing verifiable additional data on how an
artificial intelligence system evolves over time, if this can provide a
foundation into further research towards comparisons of human trust and
artificial intelligence trust.


\subsection{Maintaining the Integrity of the Specifications}


\section{Background}

Models in CNNs are the primary reason for opacity of neural networks. Decision
trees and algortihms have the benefit of being easily repeatable outside of the
system. The complexity of even the simplest models makes this unviable. 
The introduction usually describes the background of the project with brief information on general knowledge of the subject. This sets the scene by stating the problem being tackled and what the aims of the project are.

\subsection{The Role of Models in CNNs}
\subsection{Identifying Differences in Models}
\subsection{Weights and Biases}
\subsection{}
\subsection{}
\subsection{}
\subsection{}


\section{Method}
Method should outline how the task/experiment was carried out, including rationale for any decisions made. Details of any equipment and subjects used should be also included. Basically, you should include enough information, so that the reader could duplicate as much of the experimental conditions or design details as possible.

\subsection{Reducing Model without Losing Character}
The first part of the delivery required that a model, in this case a CNN model in Keras, could be reduced to an information block that each layer contains a summary of the layer that can record any changes in the weights and biases. The structure of a model is an 
\begin{enumerate}
\item input shape
\item series of dense layers
\item output layers
\end{enumerate}

and within each of the layers, there are is a series of layers that, depending on type, will contain a series of tuples that have a single of multi-dimensional series of weights, plus a set of biases. it is within these layers that we will extract the information. As there is potentially a lot of information in the layer, we attempted to reduce this by extracting two identifying elements

\begin{table}[h]
    \caption{Values generated from evaluated model}
    \setlength\tabcolsep{0pt} % make LaTeX figure out intercolumn spacing
    \begin{tabular} { p{0.25\linewidth}  p{0.7\linewidth} }
        Mean    & The average value of all layers across all dimension \\ \hline 
        StdDev  & An indication of the deviation of values, to assist in seeing how the values are weighted \\ \hline
        Histogram & A Histogram of the distribution of values across the layer. This will assist in giving a more informed distribution. As there are a large number of potential entries in the dense layers, from binary to minute changes in float values, the Histogram will be split between the values smallest non-zero granularity and the largest value across 10 bins. \\ \hline
        Name.   & Name of Layer \\ \hline
        Angle of Data & This is a single value representing the weighting of a Hash of the Layer Values (weights and biases) \\ 
    \end{tabular}
\end{table}

\subsection{Recording Changes}
As we now have a method of identifying the model in a broken down format, the next stage is to record and apply a signature to the model. There are a number of methods of ensuring that the model is recorded. The key part is to take the series of hashes on each layer, combined with the model shape, and generate a signature. The signature is what will be recorded as an immutable value that can be regenerated every time the \textbf{exact} same model is used. With this identification, we can create a list showing the parent and child of the model.

To ensure this is done correctly, we have two applications - local and global.
\subsection{Global Changes}
For global changes there will be a globa repository containing the signature and summary of published models. 

\begin{table}[h]
\caption{Model Reference Storage}

\begin{tabular}{| p{0.25\linewidth} | p{0.7\linewidth} |}
    \hline
    Model Identifier    & This is the checksum generated by the model \\
    \hline
    Creation Date       & This will be the timestamp that the model was first added to the repository \\
    \hline
    Parent              & The model identification of the direct parent of the model. \\
    \hline
    Model Summary       & A binary object that contains the items generated in the table above. \\
    \hline 
    
\end{tabular}
\end{table}

\subsection{Local Changes}

An additional benefit of the summarisation is to be able to view the model as it is being trained. This is a local filestore or database that contains a more regular record of the the summary changes. It is recommended that global model reporistory is only used for publishing models that require tracability, and not for regular changes.

In keeping with Git vocabulary, local changes are updated with the \verb|model_add(model, [parent])| command, which takes a model and returns a signature and commits the information to a local repository.

The local repository stores the abstracted model information set that can be used for local comparisons.
To commit the model to the public repository, the \verb|model_add(model, [parent])| command is used, with this time the original model that already exists in the repository. The command will return an error if the parent does not exist.


\subsection{Comparing Models}
To ensure there is tracibility in between the models, if a parent model is defined, a \textit{b-score} is generated with a degree of difference between the models. The score is based on two factors, the difference between the model structure, and the difference between the two primary parameters for each layer: \textit{StdDev} \& \textit{Angle of Array}

\subsection{Collision Management}

{ \itshape
In computer science, a collision or clash is a situation that occurs when two distinct pieces of data have the same hash value, checksum, fingerprint, or cryptographic digest.[1]

Due to the possible applications of hash functions in data management and computer security (in particular, cryptographic hash functions), collision avoidance has become a fundamental topic in computer science.

Collisions are unavoidable whenever members of a very large set (such as all possible person names, or all possible computer files) are mapped to a relatively short bit string. This is merely an instance of the pigeonhole principle.[1]

The impact of collisions depends on the application. When hash functions and fingerprints are used to identify similar data, such as homologous DNA sequences or similar audio files, the functions are designed so as to maximize the probability of collision between distinct but similar data, using techniques like locality-sensitive hashing.[2] Checksums, on the other hand, are designed to minimize the probability of collisions between similar inputs, without regard for collisions between very different inputs.[3]
}


\section{Results and Discussion}
This section includes presentations of the data and results in graphical and/or tabular form. Statistical analysis of the variation in measurements/results can also be presented. The findings should be explicitly related to the stated objectives of the project. Details and tasks that that may distract the reader unnecessarily should be presented in Appendices.

This is the section where you have a chance to discuss the results and draw conclusions from them - i.e. where you show your ability to think analytically about your findings.



\section{Conclusions and Recommendations}

The conclusion of the \textit{Practicum} is there is currently no simple and efficient method of verifying the lifecycle of the development of the model. When models are released, either for production, competitions or uploaded to critical systems, they remain a blackbox. While developing a system for verifying the history of the model, it will allow us to both gain an insight into the development process, but also be able to check shared characteristics between machine learning deep learning systems.

We explored a number of novel elements to provide more information about a model that can be used to identify differences between models, and the difference between layers of models. The next phase of the research will be to develop a study based on real world use in a development environment. The practical application yielded more data than originally envisaged, particularly in relation to the transferability of the modules to all deep neural networks.

Visualisation applications were limited to graphs showcasing the difference between models. It was not necessary to explore 3D comparison primarily because the layer information was reduced to two variables. While they have the capacity to continuously update an visual family tree while the system is in training, a direct \textit{tensorboard} style application may be more beneficial .

\subsection{Recommendations}


Our experimentation into the personality of a neural network model shows that there is additional scope for further research using these finding within the scope of interpretability and explainability. Comparisons between training models error rates and the difference in layers provide an indication into how training is successful, although there was an interesting observation where some layers were trained \textit{more}, that is the similarity between layers changed out of sync with error rates.

There are several promising spillover activities that can follow this approach. The majority are related to explainability, although a more pertinent issue will be utilisation of the techniques in this paper to a practical application for real world auditing of models as they are being trained and released. In addition, the following areas could increase knowledge of uptake of explainable AI.

\begin{enumerate}
    \item 
    Increasing information stored. 
    Creating the information blocks at a more granular level to generate the checksum with all the information currently stored can provide more utility. One the other hand, this would create a non-optimal storage if we included the full set of model weights. There may be a compromise to provide more visibility, while retaining unique characteristics of the model.
    
    \item 
    Introduction of blockchain verification.
    Currently using database solutions that can be accessed publicly. Integrating a distributed ledger to store the public models may provide more security in the models. Until, it is required, the one-way hashing algorithm provide security that the model used is identical, or closely related, to a signature.
    
    \item 
    Closer integration with Machine Learning providers.
    Including push functionality into the training and deployment process, as well as the inclusion of visualisation tools, could increase uptake and provide more sources to compare.
    
    \item 
    Testing in a model distribution environment.
    The solution can be used in an environment where pre-trained of default models can be verified prior to manipulation. This can be used to establish a metric of similarity between models (the functions have been included) once trained.
\end{enumerate}




\bibliographystyle{IEEEtran}
\bibliography{bibliography}


\end{document}

